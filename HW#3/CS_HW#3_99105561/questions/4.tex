\subsection*{الف}
در روش
LCG
برای تولید اعداد تصادفی ابتدا به سه پارامتر نیاز داریم. این پارامترها شامل بذر یا
seed
می‌باشد که یک عدد اولیه به عنوان ورودی به الگوریتم می‌دهیم تا به عنوان نقطه شروع برای تولید اعداد تصادفی مورد استفاده قرار گیرد.

همچنین یک مقدار
$a$
برای ضریب تولید اعداد تصادفی و یک مقدار
$c$
که بیانگر مقدار جابه‌جایی است خواهیم داشت.

با استفاده از رابطه زیر اعداد تصادفی تولید می‌شوند.
\[
x_{i + 1} = (a x_{i} + c) \,\, \text{mod} \,\, m    
\]
که در اینجا
$x_{i + 1}$
نشان‌دهنده عدد تصادفی جدید تولید شده براساس عدد قبلی با استفاده از باقی‌مانده تقسیم عدد بر
$m$
است.

همچنین برای شروع مقدار اولیه
$x_0$
را انتخاب کرده و با قرار دادن در ربطه دنباله اعداد تصادفی را به دست می‌اوریم.

\subsection*{ب}
مفهموم
\LR{Random-number streams}
به مجموعه‌ای از اعداد گفته می‌شود که به صورت تصادفی با استفاده از یک مولد تولید شده‌اند. در واقع به این معناست که هر جریان از اعداد شامل یک توالی از اعداد تصادفی است که می‌توان از آنها به صورت‌های استفاده کرد.

کاربرد اصلی این جریان وجود یکسری اعداد تصادفی از پیش تعیین‌شده برای کارکردهای مختلف است.

\subsection*{ج}
برای آنکه بتوان به ماکزیموم مقدار اعداد تصادفی تولید شده در الگوریتم
LCG
برسیم نیاز داریم که پارامتر‌های
$(m, c, a)$
را به این صورت مورد استفاده قرار دهیم.
\begin{itemize}
    \item
    مقدار
    $m$
    می‌بایست یک عدد اول و بزرگتر از مقادیر
    $a$
    و
    $c$
    باشد. دلیل آن این است که هر چقدر این عدد بزرگتر باشد دامنه اعداد تولید شده بیشتر می‌شود و با اول بودن این عدد می‌توان مطمئن شد که تمامی اعداد در بازه باقی‌مانده تولید شوند.

    \item
    دو عدد
    $m$
    و 
    $c$
    باید نسبت به هم اول باشند چراکه اگر نباشند شاهد تکرار یکسری اعداد در دنباله مدنظرمان خواهیم بود.

    \item
    همچنین اگر یک عامل اولیه بزرگتر از چهار برای عدد
    $m$
    داشتیم، یعنی عددی که 
    $m$
    بر آن بخش‌پذیر باشد در این صورت
    $a - 1$
    می‌بایست بر این عامل نیز بخش‌پذیر باشد.
\end{itemize}

\subsection*{د}
از مزایای روش
LCG
می‌توان به سادگی پیاده‌سازی، سرعت بالا و قابلیت تکرار پذیری اشاره کرد. این روش از این جهت ساده و سریع بوده که صرفا اعداد را براساس یک فرمول ساده می‌سازد و در هر مرحله اگر از یک بذر یکسان استفاده کنیم، دنباله مشابه خواهیم داشت.

از معایب این روش می‌توان به تکرار اعداد تصادفی در دوره‌های خاص، کمبود اعداد تصادفی ساخته‌شده و عدم انطباق با توزیع‌های مختلف اشاره نمود. از آنجایی که الگوریتم براساس باقی‌مانده بر عدد
$m$
بنا شده بدیهتا یک دوره تکرار خواهیم داشت که باعث می‌شود محدوده اعداد تصادفی ما به اندازه کافی بزرگ نباشد. همچنین این دنباله معمولا در توزیع خطی قرار گرفته که ممکن است برای تمامی کارها مناسب نباشد.