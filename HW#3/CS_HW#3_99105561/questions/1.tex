\subsection*{الف}
در این قسمت برای آنکه احتمال این را به دست آوریم که یک تسک هنگام رسیدن به سیستم منتظر بماند کافی است احتمال پر بودن سرور را به دست آوریم. برای اینکار خواهیم داشت:
\[
P(t = 0) = (\frac{\lambda}{\mu})^m
\]
که در آن
$t$
نشان‌دهنده تعداد سرورهای باقی‌مانده است.

\subsection*{ب}
تقاوت این قسمت با قسمت قبل وجود تسک در قسمت بافر است. از آنجایی که در قسمت قبل این احتمال را به صورت
$(\frac{\lambda}{\mu})^m$
به دست آوردیم. از آنجایی که ظرفیت بافر
$m$
بوده در این قسمت قرقی نخواهیم داشت وا احتمال پر بودن سرور به همان صورت خواهد بود.

\subsection*{ج}
برای به دست آوردن زمان انتظار هر تسک باید زمان انتظار در صف و بافر را به صورت جداگانه به دست آورده و جمع‌کنیم.

زمان متوسط انتظار صف برابر خواهد بود با میانگین زمان برای تسک‌ها منهای میانگین زمان ورود، یعنی خواهیم داشت.
\[
\text{متوسط زمان انتظار صف} = \frac{1}{\mu - \lambda}
\]

برای بافر نیز به این صورت خواهد بود با این تفاوت که باقر ظرفیتی معادل با
$m$
دارد.
\[
\text{متوسط زمان انتظار بافر} = \frac{m}{\mu - \lambda}
\]

به این ترتیب برای زمان متوسط انتظار کل خواهیم داشت:
\[
\text{متوسط زمان انتظار کل} = \frac{m}{\mu - \lambda} + \frac{1}{\mu - \lambda} = \frac{1 + m}{\mu - \lambda}
\]

\subsection*{د}
برای به دست آوردن کمترین مقدار
$m$
خواهیم داشت:
\begin{align*}
    p(t = 0) &= (\frac{\lambda}{\mu})^m = 1 - \gamma \\
    &\implies m \log(\frac{\lambda}{\mu}) = \log(1 - \gamma) \\
    &\implies m = \frac{\log(1 - \gamma)}{\log(\frac{\lambda}{\mu})}
\end{align*}

\subsection*{ه}
برای به دست آوردن پارامترهای داده شده خواهیم داشت:
\begin{align*}
    p &= \frac{\lambda}{\mu} = \frac{50}{60} = \frac{5}{6} \\\\
    W &= \frac{1 + m}{\mu - \lambda} = \frac{1 + m}{60 - 50} = \frac{1 + m}{10} \\\\
    L &= \lambda \times W \\
    &= 50 \times \frac{1 + m}{10} = 5(1 + m)
\end{align*}