اگر بخواهیم در حوزه مهندسی کامپیوتر از کاربردهای شبیه‌سازی کامپیوتری بگوییم می‌توانیم به آزمودن سخت‌افزارها اشاره کنیم. همانطور که می‌دانیم برای ساخت یک مدار کامپیوتری هزینه‌های زیادی را باید بپردازیم. حال اگر می‌خواستیم هر بار با تست کردن به صورت تجربی این مدارها را ببندیم شامل هزینه بسیار سنگینی می‌شدیم، در صورتی که می‌توان با استفاده از شبیه‌سازی کامپیوتری تمام حالات ممکن را آزمود و در هزینه‌های صرفه‌جویی قابل توجهی انجام داد.

همچنین آزمودن ربات‌های خودکار به صورت تجربی می‌توانند پرهزینه و کم بازده باشد، در صورتی که با استفاده از شبیه‌سازی کامپیوتری می‌توان این کار را در زمانی کمتر با بازدهی بیشتر انجام داد.

\subsection*{الف}
در اینترنت اشیا برای ارتباط بین اشیا مختلف به کارگرفته می‌شود. یک سناریو شبیه‌سازی برای این حالت می‌توان یک محل کار با انواع دستگاه و وسیله متصل به اینترنت اشیا باشد. می‌توان براساس حالت‌های مختلف این محل و براساس ورودی‌های مختلف نتایج محتمله را بررسی کرد. همچنین می‌توان حالت‌های اضطراری همانند سوختن یکی از وسایل که در حالت عادی یک اتفاق پرهزینه و نامطلوب تلقی می‌شود را به راحتی شبیه‌سازی کرد.

\subsection*{ب}
اگر بخواهیم سناریویی برای شبیه‌سازی محاسبات بر پایه ابری بیان کنیم می‌توانیم ساختار یک شرکت را مثال بزنیم. در این شرکت یکسری محاسبات پرهزینه و حیاتی توسط کامپیوترهای خود شرکت انجام می‌شود. برای ارتقا توان این محاسبات می‌خواهیم از رایانش ابری استفاده کنیم. آزمون و خطا کردن در اینجا می‌تواند به شدت هزینه‌اور و خطرناک باشد زیرا که نتایج این محاسبات برای شرکت حیاتی بوده و در هر لحظه به آن احتیاج دارد. بهترین و معقولانه‌ترین روش شبیه‌سازی براساس یک مدل بوده که تمام حالات ممکن را بیازماییم و سپس این سیستم را بر شرکت سوار کنیم.

\subsection*{ج}
سیستم‌هایی که در آنها براساس یک‌سری اطلاعات در لحظه اتفاقات صورت می‌گیرد. نمونه شفافی برای نشان دادن لازمه استفاده از شبیه‌سازی در این سیستم‌هاست. زیراکه در لحظه ممکن است تعدادی اتفاقات براساس توزیع خاص رخ دهد که این سیستم‌ها می‌بایست برای تمامی این حالات آمادگی داشته باشند. از این رو یک سناریو شبیه‌سازی کمک می‌کند که براساس تمامی حالات ممکن با هزینه اندک کارایی سیستم را بیازمایم و از کارکرد درست آن اطمینان خاطر داشته باشیم.