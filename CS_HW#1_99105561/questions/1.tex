\subsection*{الف}
از اولین معایب شبیه‌سازی کامپیوتر می‌توان به سختی ایجاد و انتخاب مدل مناسب برای کار مدنظر است و این کار نیاز به یک متخصص ماهر در این زمینه دارد. راهکار این می‌تواند آن باشد که هر شخص با تمرین و کسب مهارت در این زمینه، توانایی انتخاب و بهبود یک مدل مناسب را پیدا کند.

عیب دوم وجود هزینه و زمان بسیار زیاد برای یکسری مسائل در شبیه‌سازی کامپیوتری است. راهکار این قسمت می‌تواند تغییر مدل یا انتخاب یک مدل بهینه‌تر برای آن رخداد هزینه‌ها را کاهش داد و در زمان محاسباتی نیز صرفه‌جویی کرد.

در نهایت عیب سوم وابسته بودن نتایج براساس رندومنس ورودی‌هاست. یعنی در مدل‌های شبیه‌سازی کامپیوتری ما شاهد این هستیم که ورودی‌ها به صورت رندوم به ما داده شده و گاها نتیجه‌گیری براساس آنها امری سخت تلقی می‌شود. راهکار این مشکل هم‌ می‌توان در تحلیل نمودن نتایج مدل براساس ورودی‌های مختلف و تکرار این روند باشد.

\subsection*{ب}
روش عددی و تحلیلی در سیستم‌های خطی و غیرخطی می‌توانند رفتارهای متفاوتی داشته باشند. در سیستم خطی معمولا این دو روش جواب یکسانی برای ما خواهند داشت اما در سیستم غیرخطی با روش عددی ممکن است نتایج و خروجی‌های متفاوتی داشته باشیم.

در روش‌های تحلیلی معمولا جواب به دست آمده دقیق و قابل اتکا هستند، اما در روش‌های عددی نتایج به دست آمده ممکن است دقیق نباشد و تطابقی با واقعیت نداشته باشد.

در روش عددی، نیاز به قدرت تحلیلی نداریم و بیشتر نیاز به یک توان محاسباتی همانند کامپیوتر خواهیم داشت. در صورتی که در روش تحلیلی یک فرد می‌بایست مهارت زیاد و فهم کاملی از رخداد مدنظر داشته باشد تا بتواند به درستی و دقیق آن را تحلیل کند.

\subsection*{ج}
دو سناریو را می‌توان نام برد که در آن شبیه‌سازی بر آزمودن ارجحیت دارد. در حالاتی که آزمودن برای ما زمان بسیار زیادی داشته باشد همانند رشد یک گیاه و حالاتی که هزینه بسیار بالا و گرانی برای تجربه داشته باشد همانند ارسال یک ماهمواره.